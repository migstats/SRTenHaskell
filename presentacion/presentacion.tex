\documentclass{beamer}
\usepackage[spanish]{babel}
\usepackage[utf8]{inputenc}
\usepackage{fancyvrb}
\usepackage{amsmath}
\usepackage{amssymb}
\usepackage{enumerate} 
\usepackage{mathrsfs}
\usepackage{pgf,tikz}
\usepackage{listings}    

% Tema %%%%%%%%%%%%%%%%%%%%%%%%%%%%%%%%%%%%%%%%%%%%%%%%%%%%%%%%%%%%%%%%

\usetheme{Warsaw}
\usecolortheme{seahorse}
\useoutertheme{shadow}

% Título %%%%%%%%%%%%%%%%%%%%%%%%%%%%%%%%%%%%%%%%%%%%%%%%%%%%%%%%%%%%%%

\title[Sistemas de reescritura]{Sistemas de reescritura desde el punto
  de vista de la programación funcional}
\author[Miguel Ángel Porras Naranjo]{Miguel Ángel Porras Naranjo}

\institute[Dpto. Ciencias de la Computación e Inteligencia
Artificial]{Tutorizado por José Antonio Alonso y María José Hidalgo}

% Documento %%%%%%%%%%%%%%%%%%%%%%%%%%%%%%%%%%%%%%%%%%%%%%%%%%%%%%%%%%%

\begin{document}
\lstset{language=Haskell}
\frame{\titlepage}

\begin{frame}
  \frametitle{Índice}
  \tableofcontents
\end{frame}

\section{Introducción}

\begin{frame}
  \frametitle{Introducción}

  \begin{block}{}
    La \textbf{reescritura} es una técnica que consiste en reemplazar
    términos de una expresión con \textbf{términos equivalentes}.
  \end{block}
\end{frame}

\begin{frame}
  \frametitle{Introducción}
  \begin{block}{Ejemplo}
    Dadas las reglas $x * 0 \Rightarrow 0$ y $x+0 \Rightarrow x$,
    aplicadas a la siguiente expresión,

    \[x +(x*0) \longrightarrow x+0 \longrightarrow x \]
  \end{block}

  \begin{block}{Ejemplo}
    Dadas las reglas
    $\neg (P \wedge Q) \Rightarrow \neg P \vee \neg Q$ y
    $\neg (\neg P) \Rightarrow P$, reescribimos el siguiente término,

    \[\neg (\neg (\neg ((\neg P) \wedge Q))) \longrightarrow \neg
      ((\neg P) \wedge Q) \longrightarrow \neg (\neg P) \vee \neg Q
      \longrightarrow P \vee \neg Q\]
  \end{block}
\end{frame}

\begin{frame}
  \frametitle{Introducción}
  \begin{block}{}
    Los \textbf{sistemas de reescritura} se pueden aplicar a cualquier
    tipo de expresión (ya sea de lógica, ecuacional, etc) y podemos
    decidir qué reglas se pueden usar. Pero antes debemos formalizar
    como deben ser estas expresiones.
  \end{block}

  \begin{block}{Definición}
    Un \textbf{término} está construido por \textbf{variables},
    \textbf{símbolos constantes} y \textbf{símbolos de funciones}.
  \end{block}
\end{frame}

\begin{frame}
  \frametitle{Introducción}
  \begin{block}{}
    El término $f(i(e), f(x,e))$ se representa por,
    \begin{figure}[h] \centering
      \begin{tikzpicture} \node [circle,draw]{$f$} child {node
          [circle,draw]{$i$} child{node[circle,draw]{$e$}}} child
        {node [circle,draw]{$f$} child {node [circle,draw]{$x$}} child
          {node [circle,draw]{$e$}} };
      \end{tikzpicture}
    \end{figure}

    Donde $i,f$ representa un símbolo de función de aridad 1 y 2
    respectivamente, $e$ representa un símbolo constante y $x$
    representa una variable.
  \end{block}
\end{frame}

\begin{frame}
  \frametitle{Introducción}
  \begin{block}{}
    Los ejemplos anteriores se pueden escribir como términos,
    \[
      \begin{array}{rcl}
        x +(x*0) & \longrightarrow  & +(x,*(x,0)) \\ \\
        P \vee \neg Q & \longrightarrow & \vee(P, \neg(Q)) \\ \\
      \end{array} 
    \]
  \end{block}
\end{frame}

\begin{frame}[fragile]
  \frametitle{Introducción}

  Los nombres de variables son pares formados por un nombre y un
  índice.
  \begin{lstlisting}[frame=single]
type Nvariable = (String,Int)
  \end{lstlisting}

  Por ejemplo, $x_3$ se representa como \texttt{("x",3)}.

  Definimos el tipo de dato \texttt{Termino} de la siguiente manera,

  \begin{lstlisting}[frame=single]
data Termino = V Nvariable
             | T String [Termino]
             deriving (Eq, Show)
  \end{lstlisting}
  Por ejemplo,
  \begin{itemize}
  \item $x_2$ se representa por \texttt{(V ("x", 2))}
  \item $f(x_2,g(x_2))$ se representa por \texttt{(T "f" [V ("x", 2),
      T "g" [V ("x", 2)]])}
  \end{itemize}
\end{frame}

\section{Sistemas de reescritura}

\begin{frame}
  \frametitle{Sistemas de reescritura}
  \begin{block}{Definición}
    Un \textbf{sistema de reescritura de términos} es un conjunto de
    sustituciones termino a término. A estas sustituciones las
    llamaremos \textbf{reglas}.
  \end{block}
\end{frame}

\begin{frame}
  \frametitle{Sistemas de reescritura}
  \begin{block}{}
    Estos conjuntos de reglas determinan ciertas propiedades según los
    elementos que contengan. Por ejemplo,
    \begin{figure}[h]
      \centering
      \begin{tikzpicture}[sibling distance=10em]
        \node [draw]{$a^{2*0} + b*0$} child { node[draw]{$a^0 + b*0$}
          child{node[draw]{$1 + b*0$}} child{node[draw]{$a^0 + 0$}} }
        child { node [draw]{$a^{2*0} + 0$} child {node
            [draw]{$a^0 + 0$}} child {node [draw]{$a^{2*0}$}} };
      \end{tikzpicture}
    \end{figure}
  \end{block}
\end{frame}

\begin{frame}
  \frametitle{Sistemas de reescritura}
  \begin{block}{Problemas}
    \begin{itemize}
    \item ¿El árbol es finito?
    \item ¿Todas las hojas del árbol son la misma?
    \end{itemize}
  \end{block}
\end{frame}

\subsection{Terminación}

\begin{frame}
  \frametitle{Terminación}
  \begin{block}{Definición}
    Diremos que un sistema de reescritura \textbf{termina}, si no podemos
    aplicar ninguna regla más.
  \end{block}

  \begin{block}{}
    Tener un mayor número de reglas no siempre es mejor. Por ejemplo
    para las reglas, $x+0 \Rightarrow x$ y $x \Rightarrow x+0$,

    \[ x + 0 \longrightarrow x \longrightarrow x+0 \longrightarrow x
      \longrightarrow x+0 \dots \]
  \end{block}
\end{frame}


\begin{frame}
  \frametitle{Terminación}
  \begin{block}{Teorema}
    El problema de la terminación es \textbf{indecidible} para el caso
    general.
  \end{block}

  \begin{block}{}
    Se puede conseguir la decibilidad con varias hipótesis de partida.
  \end{block}
\end{frame}

\subsection{Confluencia}
\begin{frame}
  \frametitle{Confluencia}
  \begin{block}{Definición}
    Diremos que un sistema de reescritura es \textbf{confluente}, si
    sea cual sea el orden en el que apliquemos las reglas, llegamos al
    mismo resultado.
  \end{block}

  \begin{block}{}
    Para analizar si un sistema es confluente o no, debemos estudiar el
    momento en el que se crea una ramificación del árbol. La idea
    intuitiva es ver si la sustitución que se hace en cada rama se
    solapa entre sí. A estos puntos los llamaremos \textbf{pares
    críticos}.
  \end{block}
\end{frame}

\begin{frame}[fragile]
    \frametitle{Implementación de los pares críticos}
\begin{lstlisting}[frame=single]
parCritico :: (Termino -> Termino)
           -> (Termino, Termino)
           -> (Termino, Termino)
           -> [(Termino, Termino)]
parCritico c (l1, r1) (l2, r2)
    | sigma1 == Left UNIFICACION = []
    | otherwise = [(sigma r1, sigma (c r2))]
    where sigma1 = unificacion l1 l2
          sigma = aplicaTerm (elimR(sigma1))
          elimR (Right a) = a
\end{lstlisting}
 
\end{frame}
\subsection{Completación}

\begin{frame}
  \frametitle{Completación}

  \begin{block}{Definición}
    Podemos modificar el conjunto reglas de los reescritura para que
    se verifiquen estas dos propiedades. A este proceso se le llama
    \textbf{completación}. Los cambios que se realizan al conjunto de
    reglas van desde eliminar alguna regla, hasta añadir propias.
  \end{block}
  
\end{frame}

\begin{frame}
  \frametitle{Algoritmo de completación}

  \textbf{Input:} Un conjunto finito $E$ de $\Sigma$-identidades y un
  orden de reducción > en $T(\Sigma, V)$.

  \textbf{Output:} Si el procedimiento termina; un sistema de
  reescritura $R$ finito y convergente que es equivalente a $E$. En
  caso contrario devuelve un \texttt{Error}.

\end{frame}

\begin{frame}
  \frametitle{Algoritmo de completación}

\textbf{Inicialización:}
Si existe una identidad $(s \equiv t) \in E$ tal que
$s \not = t, s \not > t$ y $t \not > s$, entonces termina con
\texttt{Error}.
En otro caso,
$i := 0$
$R_0 := \{l \rightarrow r | (l \equiv r) \in E \cup E^{-1} \wedge l > r \}$.

\texttt{Mientras} $R_i \not = R_{i-1}$

$R_{i+1} := R_i$;

Para todo $\langle s, t \rangle \in$ \texttt{ParesCriticos($R_i$)}:

$>$ Reducir $s,t$ a forma normal $s', t'$

$>$ Si $s' \not = t'$, $s' \not > t'$ y $t' \not > s'$ terminar con \texttt{Error}

$>$ Si $s' > t'$ entonces $R_{i+1} := R_{i+1} \cup \{s' \rightarrow t'\}$

$>$ Si $t' > s'$ entonces $R_{i+1} := R_{i+1} \cup \{t' \rightarrow s'\}$

$i := i+1$

Devuelve $R_i$

\end{frame}

\begin{frame}
  \frametitle{Conclusiones del algoritmo de completación}
  \begin{block}{}
    Este algoritmo no es muy eficiente pues calcula los pares críticos
    cada vez que añade o elimina una regla.
  \end{block}

  \begin{block}{}
    Un mejor algoritmo para la completación es el de \textbf{Huet}.
  \end{block}
\end{frame}

\section{Sistemas utilizados}

\begin{frame}
  \frametitle{Sistemas utilizados}
  \begin{center}
    \includegraphics[width=.3\textwidth]{ubuntu.png}
    \hspace{.2cm}
    \raisebox{.1\height}{\includegraphics[width=.2\textwidth]{emacs.png}}
    \hspace{.2cm}
    \raisebox{.35\height}{\includegraphics[width=.4\textwidth]{haskell.png}}
  \end{center}

  \begin{center}
    \includegraphics[width=.3\textwidth]{git.png}
    \hspace{.2cm}
    \includegraphics[width=.4\textwidth]{github.png}
  \end{center}

  \begin{block}{}
    \begin{center}
      \url{http://github.com/migpornar/SRTenHaskell}
    \end{center}
  \end{block}
\end{frame}

\section{Bibliografía}

\begin{frame}[allowframebreaks]
        \frametitle{Bibliografía}
\nocite{*}
\bibliographystyle{plain}
\bibliography{presentacion}
\end{frame}


\end{document}

