\chapter*{Sistemas utilizados}
\addcontentsline{toc}{section}{Sistemas utilizados}

Durante la realización de este Trabajo de Fin de Grado he usado los
siguientes sistemas y paquetes. 

\begin{itemize}
\item \textbf{Ubuntu 16.04 LTS.} Instalé \textit{Ubuntu} en una
  partición que hice al disco duro de mi ordenador portátil. La
  instalación se realizó mediante la herramienta
  \href{http://www.linuxliveusb.com/} {\textit{LinuxLive USB
      Creator}}. 

\item \textbf{\LaTeX{} .} Descargué la distribución de \LaTeX{} y
  \textit{Tex Live} utilizando el \textit{Gestor de Paquetes
    Synaptic}. Debo destacar dos paquetes fundamentales que me han
  sido de gran utilidad;
  \begin{itemize}
  \item \textbf{Paquete AUCTex.} Anteriormente, solo había usado el
    editor, \textit{TeXstudio} para la creación de documentos
    \LaTeX{}, pero gracias a mi tutor José Antonio, me animé a probar
    \textit{Emacs}. Mediante el comando \texttt{M-x list-packages} de
    \textit{Emacs} me descargué este paquete desde el repositorio
    \textit{MELPA}. Gracias a \textit{AUCText} pude trabajar en
    \textit{Emacs} y obtener diversos paquetes del editor que me
    ayudaron en la edición, como \textit{flycheck} o
    \textit{flyspell}.

  \item \textbf{Paquete Tikz.} Usé este paquete para la realización de
    todas las figuras de este trabajo.
      
  \end{itemize}
 
\item \textbf{Haskell.} Usando el \textit{Gestor de Paquetes
    Synaptic}, me descargué los paquetes \textit{haskell-platform} y
  \textit{GHC}. La versión de \textit{Haskell} con la que he trabajado
  es la \textit{2014.2.0.0} y la del compilador \textit{GHC}, la
  \textit{7.10.3-7}. Además para trabajar en \textit{Emacs}, instalé
  el paquete \textit{haskell-mode} y \textit{doctest}. Este último
  automatizó la tarea de comprobar la veracidad de todos los ejemplos
  del código.

\item \textbf{Git y GitHub.} Para control de versiones he usado
  \textit{Git}. El editor \textit{Emacs} me facilitó el aprendizaje de
  este sistema mediante el paquete \textit{Magit}. Además todo el
  trabajo está subido a un repositorio
  \textit{Github}\footnote{\href{https://github.com/migpornar/SRTenHaskell}{https://github.com/migpornar/SRTenHaskell}}. La
  versión de \textit{Magit} que utilizo es la \textit{2016.12.01}.
  \end{itemize}

  
%%%%%%%%%%%%%%%%%%%%%%%%%%%%%%%%%%%%%%%%%%%%%%%%%%%
%%Apéndices
%%%%%%%%%%%%%%%%%%%%%%%%%%%%%%%%%%%%%%%%%%%%%%%%%%%

\clearpage
\addappheadtotoc
\appendix

%%% Local Variables:
%%% mode: latex
%%% TeX-master: "SRT_en_Haskell"
%%% End:
